\documentclass{prosper}
\usepackage{amsmath}
\usepackage{epsf}


\newtheorem{theorem}{\sc Theorem}
\newtheorem{lemma}{\sc Lemma}
\newtheorem{coro}{\sc Corollary}
\newtheorem{defin}{\sc Definition}
\newenvironment{proof}{\par \sc Proof.\rm}{\hspace*{\fill}$\bullet$\vspace{1ex}}
\newcommand{\my}[2]{{#1 \choose #2}}
\newcommand{\BFL}{\mathbf{L}}
\newcommand{\BFl}{\mathbf{l}}
\newcommand{\BFv}{\mathbf{v}}
\newcommand{\BFw}{\mathbf{w}}
\newcommand{\BFT}{\mathbf{T}}

%\title{On the State and Game Tree Complexity of Go}
\title{Combinatorics of Go}

\author{John Tromp}

\institution{
joint work with Gunnar Farneb\"{a}ck\\
full paper at http://tromp.github.io/go/legal.html
}

\begin{document}
\maketitle

\begin{slide}{Overview}
\begin{itemize}
\item History
\item Sample Game
\item Rules Summary
\item Computational Complexity
\item Number of Legal Positions
\item Number of Games
\item Open Problems
\end{itemize}
\end{slide}

\overlays{2}{
\begin{slide}{Quotes}
\begin{itemstep}
\item Go uses the most elemental materials and concepts -- line and circle,
wood and stone, black and white -- combining them with simple rules to generate
subtle strategies and complex tactics that stagger the imagination.
\\
Iwamoto Kaoru, 9-dan professional Go player and former Honinbo title holder
\item 
While the Baroque rules of chess could only have been created by humans,
the rules of go are so elegant, organic, and rigorously logical that
if intelligent life forms exist elsewhere in the universe,
they almost certainly play go.

    Edward Lasker, Chess International Master
%\item Monks who have a talent for it play go with women and become their lovers.
%
%    Yamaoka Genrin, Edo-period essayist
\end{itemstep}
\end{slide}
}

\begin{slide}{History}
\begin{itemize}
\item  Originating in China between 2000BC and 600BC (wei'qi)
%The origins of the game are unknown, but the oldest surviving references come
%from China in the 6th century BC.
\item spread to Japan in 7th century
\item gained popularity at imperial court in 8th century
\item played in general public in 13th century
\item founding of Go academy in early 17th century
%Early in the 17th century, the then best player in Japan, Honinbo Sansa, was
%made head of a newly founded Go academy (the Honinbo school, the first of
%several competing schools founded about the same time), which developed the
%level of playing greatly, and introduced the martial-arts style system of
%ranking players. The government discontinued its support for the Go academies
%in 1868 as a result of the fall of the Tokugawa shogunate.
\item Leibniz (1646-1716) published an article about go
%(admitting he does not know all the rules).
%http://senseis.xmp.net/?GoHistory:
%"Of particular interest may be the German philosopher and mathematician Leibniz
%(1646 to 1716) who published an entire article about go including a picture of
%a go game; however, Leibniz states that he does not know all the rules."
\item from 25 to 50 million go players in the Far East; \\
      known in Korea as baduk
\item central in Japanese manga and anime series
\end{itemize}
\end{slide}%{History}

\begin{slide}{Hikaru no Go}\
\epsfxsize=5cm\epsfbox{HnG-18-05-30.eps}
\end{slide}

\begin{slide}{Hikaru no Go}\
\epsfxsize=5cm\epsfbox{HnG-18-05-31.eps}
\end{slide}

%Through these long millennia and cultural wanderings there has been little
%alteration to the rules. No other game can make a similar claim to longevity
%and stability. And for good reason. The rules of go are so simple, yet so
%perfect, that it cannot be tinkered with without destroying it. Like a
%mathematical truth, go appears to have always been there, beyond time and
%space. It has been conjectured that if there are sentient beings on other
%planets they may also be playing go.
%Except for changes in the board size and
%starting position, Go has essentially kept the same rules since that time,
%which quite likely makes it the oldest board game still played today. It had


%In honour of the Honinbo school, whose players consistently dominated the other
%schools during their history, one of the most prestigious Japanese Go
%championships is called the "Honinbo" tournament.

\begin{slide}{Sample Game}
courtesy of C(omplete)Goban
\end{slide}

\begin{slide}{Concepts of Go}
\begin{itemize}
\item a grid of {\em points}: $\{0,\ldots,m-1\}\times \{0,\ldots,n-1\}$
\item 3 {\em colors}: $\{\mbox{empty}, \mbox{black}, \mbox{white}\}$
\item position: a mapping from points to colors
\item stone: a point colored black or white
\item string: a connected component of adjacent stones of the same color
\item liberty: empty point adjacent to a string
\end{itemize}
\end{slide}

\begin{slide}{Rules of Go}
\begin{itemize}
\item play starts on an empty board
\item on his turn, a player passes \\
      or makes a move that doesn't repeat an earlier position
\item move: placing a stone and removing libertyless strings, \\
      removal of opponent string taking precedence
\item consecutive passes end the game \\
\item a player's score is number of points he controls
\end{itemize}
\end{slide}

\overlays{4}{
\begin{slide}{Computational Complexity}
\begin{itemstep}
\item 1980: Lichtenstein and Sipser proved Go PSPACE-hard.
\item 1983: Robson showed Go with the basic ko rule to be EXPTIME-complete.
\item 2000: Tromp and Cr\^{a}\c{s}maru showed ladders to be PSPACE-complete.
\item 2002: Wolfe showed endgames to be PSPACE-complete.
\end{itemstep}
\end{slide}
}

\begin{slide}{all $2 \times 2$ positions}
\begin{center}
\epsfxsize=8cm \epsfbox{ALL2.eps}
\end{center}
\end{slide}%{all $2 \times 2$ positions}


\overlays{5}{
\begin{slide}{Number of legal positions: $L(m,n)$}
\begin{itemstep}
\item $L(2,2) = 3^4$ minus the number of illegal positions. \\
\item all $2^4$ positions with 4 stones are illegal.
%\epsfxsize=8cm \epsfbox{ILL2.eps}
\item all $8$ positions with a stone of one color bordered by two stones of
the opposite color are illegal.
\item all positions with 2 or fewer stones are legal.
\item Hence $L(2,2) = 81-16-8=57$.
\end{itemstep}
\end{slide}%{Number of legal positions: $L(m,n)$}
}

\begin{slide}{Game graph: $G(m,n)$ }
\begin{itemize}
\item vertices: the legal $m\times n$ positions
\item edges: moves between different positions
%Note that we remove all self-loops, which correspond to single-stone
%suicides.
\item $G(3,1)$:\\
      {\epsfxsize=9cm \epsfbox{G31.eps}}
\end{itemize}
\end{slide}%{Game graph}

\begin{slide}{Game graph $G(2,2)$}
\begin{center}
\epsfxsize=9cm \epsfbox{G22.eps}
\end{center}
\end{slide}

%\overlays{5}{
%\begin{slide}{Basic Lemmas}
%\begin{itemstep}
%\item Outgoing edges from a position are in 1-1 correspondence with the moves
%that are not single-stone suicides.
%\item Proof:
%\item When player $c$ moves at $(x,y)$ in position $p$ to reach position $q$
%\item either this is not a suicide,
%      and $q$ has one more stone of color $c$, at $(x,y)$
%\item or it is suicide,
%      and $q$ only differs from $p$ in having fewer stones of color $c$,
%      and $(x,y)$ was the last liberty of these stones in position $p$.
%\item In both cases the edge $(p,q)$ uniquely identifies the move.
%\end{itemstep}
%\end{slide}%{A useful Lemma...}
%}
%
\overlays{2}{
\begin{slide}{Basic Lemmas}
\begin{itemstep}
%\item
%A node with $k$ empty points has outdegree at most $2k$
%\item
%Each edge has an implied black or white color.
%\item
%Recall that a {\em simple} path is one that has no
%repeated vertices.
%\end{itemstep}
%\end{slide}
%}

%\overlays{8}{
%\begin{slide}{Games are paths}
%\begin{itemstep}
\item Go games are in 1-1 correspondence with simple paths \\
starting at the all-empty node in the game graph.
%\item Proof:
%\item Path; sequence of moves, not necessarily alternating in color.
%\item Insert single pass before every out-of-turn move,
%\item Append 2 passes at the end
%\item produces proper start, alternation and ending.
%\item path is simple: each move is legal.
%\item gives bijection
%Furthermore, since any game can be stripped of its
%passes to produce the corresponding path, this is a bijection.
%\end{itemstep}
%\end{slide}
%}
%
%\overlays{4}{
%\begin{slide}{Connectedness}
%\begin{itemstep}
\item The game graph is strongly connected.
%\item Proof:
%\item any position reachable from all-empty node
%\item repeated play on liberty till suicide reaches all-empty node
\end{itemstep}
\end{slide}
}

\overlays{3}{
\begin{slide}{Counting legal positions}
\begin{itemstep}
\item Brute force; test each of $3^{mn}$ positions for legality
\item limited to counting up to $5 \times 5$.
\item Dynamic Programming; computes solution from stored
      solutions of smaller instances.
\end{itemstep}
\end{slide}
}

\begin{slide}{Partial Boards}
\begin{itemize}
\item partial go board up to column $x$ and row $y$ \\
      consists of all the points to the left of and above $(x,y)$
\item partial $7\times 7$ positions up to $(3,3)$:
{\epsfxsize=10cm \epsfbox{partial.eps}}
\end{itemize}
\end{slide}

\begin{slide}{Border States}
\begin{itemize}
\item the board height $m$,
\item the size $0 \leq y <m $ of the partial column,
\item the color of border points
      $(x,0),\ldots,(x,y-1),(x-1,y),\ldots,(x-1,m-1)$,
\item for each stone on the border, whether it has liberties,
\item connections among libertyless stones.
\end{itemize}
A state with partial column size $y$ is called a $y$-state.
\end{slide}

%A state with height $m$ and partial column size $y$ is called an
%$\my{m}{y}$-state,
%or simply $y$-state if $m$ is clear from context.
%A state is called {\em constructible} if it is the border state of some
%partial board.
%\end{defin}

%Note that $x$ acts only as a symbol whose value is immaterial to the state.
%In the figure, libertyless stones and their connections are indicated with
%lines emanating to the left.

%\begin{defin}[state counts]
%For an $\my{m}{y}$-state $s$ and $n\geq 1$,
%denote by $L(m,n,y,s)$ the number of
%partial boards up to $(n,y)$ that have border state $s$
%and all of whose libertyless stones are on, or connected to, the border.
%Call a $y$-state $s$ {\em legal} if $y=0$ and $s$ has no libertyless stones.
%\end{defin}
%
%Obviously, we have
%
%\begin{lemma}[color symmetry]
%\label{color-symmetry}
%Let state $s'$ be derived from state $s$ by reversing the colors
%of all stones. Then $L(m,n,y,s) = L(m,n,y,s')$.
%\end{lemma}
%
%\begin{defin}[state classes]
%We denote by $[s]$ the equivalence class of state $s$ under color reversal.
%Call a class {\em legal} when its members are,
%and define $L(m,n,y,[s]) = \sum_{s' \in [s]} L(m,n,y,s')$.
%\end{defin}
%
%Note that all equivalence classes, except for all-empty states,
%consist of exactly 2 states.
%These definitions immediately imply
%
%\begin{lemma}
%$L(m,n) = \sum_{\mbox{legal }s} L(m,n,0,s)
%        = \sum_{\mbox{legal }[s]} L(m,n,0,[s])$.
%\label{legal-states}
%\end{lemma}
%
%Another form of symmetry occurs in $0$-states only:
%
%\begin{lemma}[up-down symmetry]
%\label{up-down-symmetry}
%Let state $s'$ be derived from $0$-state $s$ by
%reversing the order of points from top to bottom.
%Then $L(m,n,y,s) = L(m,n,y,s')$ and $L(m,n,y,[s]) = L(m,n,y,[s'])$.
%\end{lemma}
%
%The set of constructible states is difficult to characterize,
%and hence to count. We therefore introduce a slightly larger class.
%
%\begin{defin}
%Call a $y$-state $s$ {\em valid} if it satisfies all the following:
%\begin{itemize}
%\item empty points on the border imply liberties for adjacent stones.
%\item adjacent same-colored stones are connected.
%\item connections form an equivalence relation.
%\item connections don't cross, i.e. if 4 stones are ordered vertically as
%      $a,b,c,d$, with $a$ and $c$ connected, and $b$ and $d$ connected,
%      then they must all be connected.
%\item if a stone at $(x,y-1)$ either
%\begin{itemize}
%\item has connections, but (if $y>1$) not to $(x,y-2)$, or
%\item has   liberties, but (if $y>1$) $(x,y-2)$ is opposite-colored,
%\end{itemize}
%      then points $(x,y-1)$ and $(x-1,y)$ are considered adjacent.
%\end{itemize}
%\end{defin}
%
%\begin{lemma}
%\label{constructibleisvalid}
%Every constructible state is valid.
%\end{lemma}
%
%\begin{proof}
%The first three properties obviously hold.
%The non-crossing property can be seen to follow from the planarity
%of two-dimensional boards, while the last condition holds because
%border point $(x,y-1)$ must get its liberties/connections through
%$(x-1,y-1)$, which is adjacent to $(x-1,y)$.
%\end{proof}
%
%We can efficiently compute the number of valid states using Dynamic
%Programming on partial border states, described by the stack of colors
%of pending connections and the type of the last point. Computing
%the number of $\my{19}{0}$ states in this manner takes only 0.005 sec.
%Table~\ref{borderstates} shows for each $m$ the minimum and maximum
%number of valid $\my{m}{y}$-state classes,
%which turn out to be achieved at $y=0$ and $y=m-1$, respectively.
%The maximum is always about 45\% larger than the minimum.
%
\begin{slide}{Number of State Classes}
\begin{table}
\begin{tabular}{|r|l|l|}
\hline
$m$ & \#valid $0$-classes & \#valid $(m-1)$-classes \\ \hline
1   & 3 & 3 \\
2   & 9 & 13 \\
3   & 32 & 46 \\
5   & 444 & 642 \\
7   & 6742 & 9808 \\
9   & 109736 & 160286 \\
11  & 1894494 & 2772774 \\
13  & 34320647 & 50258461 \\
15  & 645949499 & 945567689 \\
17  & 12526125303 & 18320946269 \\
19  & 248661924718 & 363324268018 \\
%29  & 954288877719528758 & 1387841561460467776 \\
%39  & 4473796211426669492286473 & 6487684033510851922982593 \\
%49  & 23280658008873608262573408056028 & 33699888311500814844062557229102 \\
%59  & 129235153940653928001315951386080547675 & 186846769643959178001661457280461585567 \\
\hline
\end{tabular}
\label{borderstates}
\end{table}
\end{slide}%{Number of State Classes}

\begin{slide}{The border state graph $B(m)$}
\begin{itemize}
\item successor states: {\epsfxsize=9cm \epsfbox{expandborder.eps}}
\item vertices: the constructible states of height $m$
\item edges: from each $y$-state to its 2 or 3 successor $((y+1)\bmod m)$-states.
\end{itemize}
%This is essentially the Map operation in Map-Reduce.
%The Reduce part sums the counts of the predecessors of a state.
\end{slide}%{The border state graph $B(m)$}

\begin{slide}{$AB(1)$}
{\epsfxsize=10cm \epsfbox{B1.eps}}
\end{slide}%{The border state graph $B(m)$}

\begin{slide}{$B(2)$}
{\epsfxsize=10cm \epsfbox{B2.eps}}
\end{slide}%{The border state graph $B(m)$}

\begin{slide}{Positions are Paths}
Legal $m \times n$ positions are in 1-1 correspondence with paths
of length $mn$ through $AB(m)$ starting at the all-Edge $0$-state
and ending at a $0$-state with no libertyless stones. \\
{\epsfxsize=10cm \epsfbox{Bpath.eps}}
\end{slide}%{paths in border state graph}

%\begin{lemma}
%\label{connected-border-graph}
%  The border state graph is strongly connected.
%\end{lemma}
%
%\begin{proof}
%Every state reaches the all-empty $0$-state within $2m-1$ empty
%successors. From the latter, we can reach any possible column~0 state $s$,
%and hence any constructible state, as follows. Note that $s$ can be reached
%in $m$ steps from a $0$-state $s'$ that replaces each stone in $s$ by
%a stone with liberties of the opposite color, effectively forming
%a virtual edge. Any such state $s'$ can clearly by reached from
%the all-empty $0$-state in $m$ steps.
%\end{proof}
%
%\subsection{Recurrences}
%
%\begin{defin}[state count vector]
%Denote by $\BFL(m,n,y)$ the state-indexed vector with elements
%$L(m,n,y,s)$ for all constructible $y$-states $s$, and by $\BFl_m$ the
%characteristic vector of legal states of height $m$.
%\end{defin}
%
%Now Lemma~\ref{legal-states} can be expressed as
%\[L(m,n) = \BFl_m^T \BFL(m,n,0).\]
%The following crucial observation forms the basis for the recurrences
%we derive.
%Since $L(m,n,y+1,s')$ can be written as a sum of
%$L(m,n,y,s)$ over all predecessor states $s$ of $s'$ in the border
%state graph, it follows that
%the border state vectors $\BFL(m,n,y)$ are related by linear
%transformations $\BFT_{m,y}$, such that
%$\BFL(m,n,y+1) = \BFT_{m,y} \BFL(m,n,y)$
%(we abuse notation by taking $\BFL(m,n,m)$ as a synonym for $\BFL(m,n+1,0)$).
%Indeed, the $\BFT_{m,y}$
%appear as submatrices of the transposed adjacency matrix of
%the border state graph.
%As a consequence, successive $0$-state vectors are related as
%
%% \label{transformation-matrices}
\begin{slide}{State count vectors}
\begin{itemize}
\item consider the board height $m$ fixed.
\item $\BFL(n,y)$: vector containing partial board count for all
possible border states $s$
\item $\begin{array}{cccc}
\BFL(0,0) & \BFL(1,0) & \BFL(2,0) & \BFL(3,0) \\
\BFL(0,1) & \BFL(1,1) & \BFL(2,1) & \\
\BFL(0,2) & \BFL(1,2) & \BFL(2,2) &
\end{array}$
\item border state graph yields linear transformations $\BFT_{y}$
\item such that $\BFL(n,y+1) = \BFT_{y} \BFL(n,y)$
\item hence $\BFL(n+1,0) = \BFT_{m-1} \BFT_{m-2} \dots
        \BFT_{1} \BFT_{0} \BFL(n,0)$
\end{itemize}
\end{slide}

%Thus we are led to define

\begin{slide}{Recurrences}
\begin{itemize}
\item gives a matrix power expression for $L(m,n)$: \\
  $L(m,n) = \BFl^T \BFT^n \BFL(n,0)$
\item where $\BFT=\BFT_{m-1} \dots \BFT_{0}$,
\item $\BFL(n,0)$ is a unit vector for the all-Edge state,
\item and $\BFl$ is the characteristic vector of legal states.
\end{itemize}
\end{slide}

%Furthermore, $L(m,n)$ can be shown to satisfy a recurrence not
%involving the border state counts. To simplify the following
%derivations, $m$ is understood to be fixed and is dropped from the
%notation, so that $L(m,n) = \BFl^T \BFT^n \BFv(0)$.

%\begin{theorem}
%\label{recurrence}
%  For fixed $m$, $L(m,n)$ satisfies a linear recurrence
%  whose order is at most the number of valid $0$-states.
%\end{theorem} 

%\begin{slide}{Recurrence without state counts}
%\begin{itemize}
%\item $p(\lambda)$: characteristic polynomial of matrix $\BFT$,
%\item $p(\lambda) = \det (\lambda \mathbf{I} - \BFT) =
%    \lambda^r + a_{r-1} \lambda^{r-1} + \dots + a_1 \lambda + a_0$
%\item Cayley-Hamilton theorem says $p(\BFT) = \mathbf{0}$:
%\item $\BFT^r = -(a_{r-1}\BFT^{r-1} + \dots + a_1 \BFT + a_0 \mathbf{I})$
%\item multiplying by $\BFT^k$, then by $\BFl^T$ on the left and $\BFv(0)$ on the right yields
%\item $L(m,k+r) = \BFl^T \BFT^{r+k} \BFv(0)
%      = -(a_{r-1} L(m,k+r-1) + \dots + a_1 L(m,k+1) + a_0 L(m,k))$
%\end{itemize}
%\end{slide}

%\begin{slide}{Closed form}
%\begin{itemize}
%\item
%  For fixed $m$, $L(m,n)$ can be written in the form
%    $L(m,n) = \sum_k q_k(n) \lambda_k^n$
%\item
%  $\lambda_k$: distinct eigenvalues of $\BFT$
%\item $q_k$: polynomials of degree $\leq$ multiplicity of $\lambda_k$ minus 1. 
%\item Perron-Frobenius theorem guarantees unique largest eigenvalue:
%\item $L(m,n)=\Theta(\lambda_m^n)$
%\end{itemize}
%\end{slide}

%Notice that some of the terms may vanish but not the largest
%eigenvalue, which we have additional information about.
%
%\begin{theorem}
%  For large $n$, $L(m,n)=\Theta(\lambda_m^n)$ for some real
%  $\lambda_m$ with $0<\lambda_m < 3^m$.
%\end{theorem}
%
%\begin{proof}
%By lemma \ref{connected-border-graph} it
%follows that $\BFT$ is regular, i.e.\ $\BFT^k$ is (elementwise)
%positive for some $k$. Since by construction $\BFT$ is also
%non-negative, the Perron-Frobenius theorem guarantees the existence of a
%real positive eigenvalue $\lambda$ with the properties that it has
%multiplicity one, is strictly larger in magnitude than
%all other eigenvalues, and has left and right
%eigenvectors $\mathbf{e}^T$ and $\mathbf{f}$ with all elements
%positive. Since $\lambda$ dominates all other eigenvalues, $\BFT^n
%\rightarrow \lambda^n \mathbf{f} \mathbf{e}^T$ for large $n$ so that
%$\BFl^T \BFT^n \BFv(0) \rightarrow (\BFl^T \mathbf{f})(\mathbf{e}^T
%\BFv(0)) \lambda^n$ and since $\mathbf{e}^T$ and $\mathbf{f}$ are
%positive this term cannot vanish. The upper limit of $\lambda$
%follows from the fact that $L(m,n)= o(3^{mn})$.
%\end{proof}
%
%The recursion coefficients are most easily obtained by
%computing $L(m,n)$ for $n=1,\dots,2s$
%by the dynamic programming algorithm in section
%\ref{DP-algorithm}. Since we know that the sequence must satisfy
%\emph{some} linear recurrence, the problem is reduced to determining the
%minimal order and the corresponding coefficients. Moreover, we know
%that the minimal order is upper-bounded by the number of valid state
%classes $s$ given in Table~\ref{borderstates}.
%
%\begin{lemma}
%\label{coefficient-lemma}
%  Assume that the sequence $x(1), x(2), \dots$ satisfies the linear
%  recurrence
%  \[
%    x(k+r) = c_{r-1} x(k+r-1) + \dots + c_1 x(k+1) + c_0 x(k).
%  \]
%  Then the coefficients $c_i$ satisfy the equation system
%  \[
%    \begin{pmatrix}
%      x(1) & x(2) & \dots & x(r) \\
%      x(2) & x(3) & \dots & x(r+1) \\
%      \vdots & \vdots & \ddots & \vdots \\
%      x(r) & x(r+1) & \dots & x(2r-1) \\
%    \end{pmatrix}
%    \begin{pmatrix}
%      c_0 \\ c_1 \\ \vdots \\ c_{r-1}
%    \end{pmatrix} = 
%    \begin{pmatrix}
%      x(r+1) \\ x(r+2) \\ \vdots \\ x(2r)
%    \end{pmatrix}.
%  \]
%  Furthermore, $r$ is the minimal order of recurrence iff
%  the above matrix is non-singular, in which case the coefficients
%  are uniquely determined.
%\end{lemma}
%
%As we do not know the minimal order of recurrence, we form
%matrices for increasing $r$. For each non-singular one, we
%compute the recurrence coefficients and verify the recurrence for
%all $2s$ elements. The smallest verifiable $r$ is the minimal order.
%
%\subsubsection{$1 \times n$ Boards}
%\label{1xn-boards}
%
%For one-dimensional boards, with $m=1$, Figure~\ref{onebordergraph} shows
%the five possible border
%states ``empty'', ``black with liberty'', ``white with liberty'',
%``black without liberty'', and ``white without liberty''. The first
%three are legal, so $\BFl = (1,1,1,0,0)^T$. The state count
%transformation is given by the transposed adjacency matrix
%\[
%  \BFT =
%  \begin{pmatrix}
%    1 & 1 & 1 & 1 & 1 \\
%    1 & 1 & 0 & 0 & 0 \\
%    1 & 0 & 1 & 0 & 0 \\
%    0 & 0 & 1 & 1 & 0 \\
%    0 & 1 & 0 & 0 & 1
%  \end{pmatrix}
%\]
%and the initial state count with one column gives $\BFv(1) = (1,0,0,1,1)^T$.
%It follows that $\BFv(0) = \BFT^{-1} \BFv(1) = (-1,1,1,0,0)^T$ and
%\[
%  L(1,n) = \BFl^T \BFT^n \BFv(0) =
%  \begin{pmatrix}
%    1 & 1 & 1 & 0 & 0 
%  \end{pmatrix}
%  \begin{pmatrix}
%    1 & 1 & 1 & 1 & 1 \\
%    1 & 1 & 0 & 0 & 0 \\
%    1 & 0 & 1 & 0 & 0 \\
%    0 & 0 & 1 & 1 & 0 \\
%    0 & 1 & 0 & 0 & 1
%  \end{pmatrix}^n
%  \begin{pmatrix}
%    -1 \\ 1 \\ 1 \\ 0 \\ 0 
%  \end{pmatrix},
%\]
%which gives the sequence $1,5,15,41,113,313,867,2401,6649,18413,\dots$
%
%The characteristic polynomial of $\BFT$ is
%$p(\lambda)=\det(\lambda \mathbf{I} - \BFT) =
%\lambda^5 - 5\lambda^4 + 8\lambda^3 - 6\lambda^2 + 3\lambda - 1$. It
%follows that $L(1,n)$ satisfies the recurrence
%\[
%  L(1,k+5) = 5 L(1,k+4) - 8 L(1,k+3) + 6 L(1,k+2) - 3 L(1,k+1) + L(1,k).
%\]
%This is not a minimal order recurrence, however. Using state classes
%instead yields
%\begin{align*}
%  \BFl &=
%  \begin{pmatrix}
%    1 & 1 & 0
%  \end{pmatrix}^T, \\
%  \BFT &=
%  \begin{pmatrix}
%    1 & 1 & 1 \\
%    2 & 1 & 0 \\
%    0 & 1 & 1 \\
%  \end{pmatrix},\\
%  \BFv(1) &=
%  \begin{pmatrix}
%    1 & 0 & 2
%  \end{pmatrix}^T
%\end{align*}
%and a characteristic polynomial $p(\lambda) = \lambda^3 - 3 \lambda^2
%+ \lambda - 1$, leading to the minimal recurrence
%\[
\overlays{3}{
\begin{slide}{Small dimensional boards}
\begin{itemstep}
\item $L(1,k+3) = 3 L(1,k+2) - L(1,k+1) + L(1,k)$ \\
  $(\lambda_1 = 2.769)$
\item $L(2,n+7) = 10 L(2,n+6) - 16 L(2,n+5) + 31 L(2,n+4) -13 L(2,n+3) +
    20 L(2,n+2) + 2 L(2,n+1) - L(2,n)$ \\
   $(\lambda_2 = 8.534)$
\item $L(3,n+19) = 33 L(3,n+18) - 233 L(3,n+17) + 1171 L(3,n+16) -
    3750 L(3,n+15) + 9426 L(3,n+14) - 16646 L(3,n+13) +
    22072 L(3,n+12) - 19993 L(3,n+11) + 9083 L(3,n+10) +
    1766 L(3,n+9) - 4020 L(3,n+8) + 6018 L(3,n+7) -
    2490 L(3,n+6) - 5352 L(3,n+5) + 1014 L(3,n+4) -
    1402 L(3,n+3) + 100 L(3,n+2) + 73 L(3,n+1) - 5 L(3,n)$ \\
    $(\lambda_3 = 25.45)$
\end{itemstep}
\end{slide}
}

\begin{slide}{$L(m,n)=a_m \lambda_m^n (1+o(1))$}
{\scriptsize
\begin{table}
  \begin{center}
    \begin{tabular}{|l|l|l|l|}
      \hline
      size & order & $a_m$ & $\lambda_m$ \\ \hline
      $ 1 \times n$ & 3   & 0.69412340909080771809 & 2.76929235423863141524 \\ \hline
      $ 2 \times n$ & 7   & 0.77605920648443217564 & 8.53365251207176310397 \\ \hline
      $ 3 \times n$ & 19  & 0.76692462372625158688 & 25.44501470555814081494 \\ \hline
      $ 4 \times n$ & 57  & 0.73972591465609392167 & 75.70934113501819973789 \\ \hline
      $ 5 \times n$ & 217 & 0.71384057986002504205 & 225.28834590398701930674 \\ \hline
      $ 6 \times n$ & 791 & 0.68921150040083474629& 670.39821492744590475404 \\ \hline
      $ 7 \times n$ & 3107 & 0.66545979340188479816 & 1994.92693537832618289977 \\ \hline
      $ 8 \times n$ & 12110 & 0.64252516474515096185 & 5936.37229306818075324832 \\ \hline
      $ 9 \times n$ & 49361 & 0.62038058380200867949 & 17665.06600837227629766227 \\ \hline
    \end{tabular}
  \end{center}
\end{table}
}
\end{slide}

\begin{slide}{The Dynamic Programming algorithm}
\begin{itemize}
\item count modulo some number $M \sim 2^{64}$
\item gives equations $L(m,n) = a_i \bmod M_i$, solvable with CRT
\item represents border state classes with 3 bits per point \\
%      , or $3m$ bits for
%a state class. This makes the standard height of $m=19$ fit
%comfortably in 64-bit integers.
%The non-crossing connections can be represented with just 2 booleans
%per libertyless stone: whether it has a connection above it, and whether
%it has a connection below it.
%The representation further exploits the fact that neighbouring points
%in the border highly constrain each other. Figure~\ref{borderdfa}
%shows possible transitions from one point to the next in a ``bump-free''
%$0$-state. Upward and downward pointing arrows from the line
%indicating lack of liberties represent the two boolean flags.

%\begin{figure}
%\begin{center}
%\epsfxsize=10cm \epsfbox{borderdfa.eps}
%\end{center}
%\caption{Intra-border transitions.}
%\label{borderdfa}
%\end{figure}
%
%Edges between boxed sets of points indicate the presence of edges from any
%point in one set to any point in the other.
%Next to each point is shown its 3-bit code. Note that no point
%has two different transitions to same numbered points. 
%This reflects the fact that two libertyless adjacent stones have the same
%color if and only if they are connected, and a stone with liberties cannot
%be adjacent to a libertyless stone of the same color.
%The algorithm uses code 0 for points in the edge to aid in the construction
%of $\BFv_m(1)$.
%Two pieces of information are still lacking; the color of a libertyless
%stone at $(x,0)$, and the color of a libertyless stone on $(x-1,y)$, which
%is not adjacent to the previous border point at $(x,y-1)$.
%However, since we represent state classes rather than states, we can
%assume that $(x-1,y)$, if non-empty, is always white.
%In this case, the color of a libertyless stone at $(x,0)$ can be stored
%in the boolean indicating connections above, since the latter is always false.
%If $(x-1,y)$ is empty, then we can assume any stone on $(x,0)$ is white.
%If both  $(x-1,y)$ and $(x,0)$ are empty then we can normalize
%the color of e.g. the bottom-most stone on the border.
%
\item computes $\BFL(n,y+1) = \BFT_{y} \BFL(n,y)$ \\
%by looping over $(s,i)$ in $\BFL(m,n,y)$ \\
%computes the 2 or 3 successors $s'$ of $s$ \\
%storing new pairs $(s',i)$ in some datastructure that facilitates sorting.
\item states are partitioned over multiple cpus so as to exhaust
available I/O bandwidth.
\item state-count pairs are stored in hundreds of individually
sorted files, which are read in parallel and merged.
\item new state-counts used to be stored in Judy array
(256-ary digital trie optimized for memory usage and speed)
mapping 64-bit keys to 64-bit values.
\item whenever memory is full, one file is written for each cpu.
\item a new cache-optimized variation on radix sort replaces the Judy trees
and makes the program run almost twice as fast.
%\item 
%From $\BFv_m(n)$, obtain $L(m,n) \bmod M$ by summing legal state counters.
\end{itemize}
\end{slide}

%\begin{slide}{Complexity}
%\begin{itemize}
%\item Space $\lambda^m(3m+64)$ bits for (state,count) pairs
%\item Time is product of
%\item \#moduli $\lceil mn \log_2(3) / 64\rceil$
%\item \#passes $mn$
%\item \#states $O(\lambda^m)$
%\item work per state $O(m)$ 
%\item total: $O(m^3 n^2 \lambda^m)$
%\end{itemize}
%\end{slide}

\begin{slide}{Results}
\tiny
\begin{table}
\label{legalcounts}
\begin{center}
\begin{tabular}{|r|l|l|}
\hline
$n$ & \#digits & $L(n,n)$ \\ \hline
1   & 1 & 1 \\
2   & 2 & 57 \\
3   & 5 & 12675 \\
4   & 8 & 24318165 \\
5   & 12 & 414295148741 \\
6   & 17 & 62567386502084877 \\
7   & 23 & 83677847847984287628595 \\
8   & 30 & 990966953618170260281935463385 \\
9   & 39 & 103919148791293834318983090438798793469 \\
10  & 47 & 96498428501909654589630887978835098088148177857 \\
11  & 57 & 793474866816582266820936671790189132321673383112185151899 \\
12  & 68 & 5777425848951323899823797030748399932728721075699118965594265 \\
    &    & 1331169 \\
13  & 80 & 3724979230768639644229490476702451767424915794820871753325479 \\
    &    & 9550970595875237705 \\
14  & 93 & 2126677329003662242497893576504405980988058610832691271966238 \\
    &    & 72213228196352455447575029701325 \\
15  & 107 & 1075146430836138311876841375486612380973378882032784440276460 \\
    &    & 1662870883601711298309339239868998337801509491 \\
16  & 121 & 4813066963822755416429056022484299646486874100967249263944719 \\
    &    & 599975607459850502222039591149331431805524655467453067042377 \\
17 & 137 & 190793889196281992046057261818504652201510583381479222439672692319440 \\
    &    & 59187214767997105992341735209230667288462179090073659712583262087437 \\
\hline
\end{tabular}
\end{center}
\label{counts}
\end{table}
\end{slide}

%\begin{slide}{Approximate Results}
%\tiny
%\begin{table}
%\label{legalcounts}
%\begin{center}
%\begin{tabular}{|r|l|}
%\hline
%$n$ & $L(n,n)$ \\ \hline
%16  & $\sim 0.035 \cdot 3^{256} \sim 4.9 \cdot 10^{120}$ \\
%17  & $\sim 0.0247 \cdot 3^{289} \sim 1.91 \cdot 10^{136}$ \\
%18  & $\sim 0.0173 \cdot 3^{324} \sim 6.6 \cdot 10^{152}$ \\
%19  & $\sim 0.01196 \cdot 3^{361} \sim 2.081 \cdot 10^{170}$ \\
%29  & $\sim 1.2 \cdot 10^{-4} \cdot 3^{841} \sim 2.2 \cdot 10^{397}$ \\
%39  & $\sim 2.4 \cdot 10^{-7} \cdot 3^{1521} \sim 1.2 \cdot 10^{719}$ \\
%49  & $\sim 9 \cdot 10^{-11} \cdot 3^{2401} \sim 3.5 \cdot 10^{1135}$ \\
%59  & $\sim 7 \cdot 10^{-15} \cdot 3^{3481} \sim 5 \cdot 10^{1646}$ \\
%69  & $\sim 1.1 \cdot 10^{-19} \cdot 3^{4761} \sim 4 \cdot 10^{2252}$ \\
%79  & $\sim 3 \cdot 10^{-25} \cdot 3^{6241} \sim 2 \cdot 10^{2953}$ \\
%89  & $\sim 2 \cdot 10^{-31} \cdot 3^{7921} \sim 4 \cdot 10^{3748}$ \\
%99  & $\sim 2 \cdot 10^{-38} \cdot 3^{9801} \sim 4 \cdot 10^{4638}$ \\
%\hline
%\end{tabular}
%\end{center}
%\end{table}
%\end{slide}
%
%\begin{slide}{Prob(random $n\times n$ position is legal)}
%\epsfxsize=10cm \epsfbox{Prob.eps}
%\end{slide}

\begin{slide}{Asymptotics}
{\epsfxsize=9cm \epsfbox{knight.eps}} \\
$3^{\frac{4}{5}n^2}(1-2/81)^{\frac{4}{5}n} \leq L(n,n) \leq 3^{ n^2  }(1-2/81)^{\frac{4}{5}n}(1-2/243)^{\frac{1}{5}n^2-\frac{4}{5}n}$
\end{slide}

\begin{slide}{Asymptotics}
\begin{itemize}
\item
Both $\lambda_m^{1/m}$ and $L(n,n)^{n^{-2}}$ converge to the same
value $L$, the {\em base of liberties}
\item
2-dimensional analogue of the
1-dimensional growth rate $\lambda_1 \sim 2.769$.
\item
Since $\frac{L(m,n+1)}{L(m,n)}$ converges to $\lambda_m$, \\
$\frac{L(m,n)L(m+1,n+1)}{L(m,n+1)L(m+1,n)}$ converges to
$\lambda_{m+1}$/$\lambda_m$, \\
which we expect to converge to $L^{m+1}/L^m=L$.
\end{itemize}
\end{slide}

\begin{slide}{Base of Liberties}
\tiny
\begin{table}
\begin{center}
\begin{tabular}{|r|l|}
\hline
$n$ & $L(n,n)L(n+1,n+1)/L(n,n+1)^2$ \\ \hline
1   & 2.28 \\
2   & 3.0 \\
3   & 2.979 \\
4   & 2.9756 \\
5   & 2.975732 \\
6   & 2.9757343 \\
7   & 2.9757341927 \\
8   & 2.9757341918 \\
9   & 2.9757341920444 \\
10  & 2.9757341920441 \\
11  & 2.975734192043350 \\
12  & 2.975734192043355 \\
13  & 2.97573419204335727 \\
14  & 2.975734192043357255 \\
15  & 2.97573419204335724932 \\
16  & 2.9757341920433572493662 \\
\hline
\end{tabular}
\end{center}
\end{table}
\end{slide}

\begin{slide}{A formula for $L(m,n)$}
\begin{itemize}
\item
$L(m,n) \approx \alpha \beta^{m+n} L^{mn}$
\item
Using the value $L=2.97573419204335725$, we can solve for $\alpha$ and
$\beta$ with the computed values of $L(15,15),L(15,16)$ and $L(16,16)$,
yielding
\item
$\alpha \approx 0.850639925845833, \beta \approx 0.96553505933836965$
\item
Achieves
relative accuracy $0.99993$ at $n=5$, $0.99999999$ at $n=9$, and
$1.00000000000025$ at $n=13$.
\item
$L(19,19) \approx 2.0816819938 \cdot 10^{170}$.
\end{itemize}
\end{slide}

\begin{slide}{Number of Games}
\begin{table}
\begin{center}
\begin{tabular}{|c||c|c|c|c|c|c|}
\hline
$m \setminus n$ & 1 & 2 & 3 & 4 & 5 & 6 \\ \hline
1 & 1 & 9 & 907 & 2098407841 & $\sim 10^{31}$ & $\sim 10^{100}$ \\
2 & & 386356909593 & $\sim 10^{86}$ & $10^{\sim 5.3\cdot 10^2}$ & & \\
3 & & & $10^{\sim 1.1\cdot 10^3}$ & & & \\
\hline
\end{tabular}
\end{center}
\end{table}
Approximate values by Heuristic Sampling.
\end{slide}

\begin{slide}{Heuristic Sampling (Pang Chen)}
\begin{tabbing}
\,\,\,\,1: \= $Q \leftarrow \{ (\it{root},1) \}$ \\
\,\,\,\,2: \> \bf{wh}\=\bf{ile} $Q$ not empty \bf{do} \\
\,\,\,\,3: \>       \> $(s,w) \leftarrow \it{pop}(Q)$ \\
\,\,\,\,4: \>       \> \bf{fo}\=\bf{r all} children $t$ of $s$ \bf{do} \\
\,\,\,\,5: \>       \>        \> $\alpha \leftarrow h(t)$ \\
\,\,\,\,6: \>       \>        \> \bf{if} \=$Q$ contains an element $(s_\alpha,w_\alpha)$ in stratum $\alpha$ \bf{then} \\
\,\,\,\,7: \>       \>        \>         \> $w_\alpha \leftarrow w_\alpha + w$ \\
\,\,\,\,8: \>       \>        \>         \> with probability $w/w_\alpha$ do $s_\alpha \leftarrow t$ \\
\,\,\,\,9: \>       \>        \> \bf{else} \\
10: \>      \>        \>         \> insert a new element $(t,w)$ into $Q$ \\
11: \> \> \> \bf{end if} \\
12: \> \> \bf{end for} \\
13: \> \bf{end while} 
\end{tabbing}
\end{slide}

\begin{slide}{Upper bounds}
\begin{itemize}
\item
On boards larger than $1\times 1$, every node in the game graph has
outdegree at least 2.
\item \#games $ \leq \prod_v \mbox{outdeg}(v) \,\,\,\,\,(mn>1)$
\item average outdegree close to $2mn/3$ in the limit
\item \#games $ \leq(mn)^{L(m,n)}$
\end{itemize}
%By Corollary~\ref{outdeg}, this is in turn bounded by $(2mn)^{L(m,n)}$.
%Most positions have about $mn/3$ empty points though, and some of the moves
%are illegal self-loops, so the average outdegree is much less than $2mn$.
\end{slide}


\begin{slide}{Lower bounds}
Suppose the $mn$ points on the board can be partitioned
into 3 sets $B,W,E$ such that
\begin{itemize}
\item $|B|=|W|=k, |E|=l=mn-2k$,
\item $B$ and $W$ are connected,
\item each point in $E$ is adjacent to both $B$ and $W$
\end{itemize}
Then there are at least $(k!)^{2^{l-1}}$ possible games,
all lasting over $k{2^{l-1}}$ moves.
\end{slide}

\begin{slide}{Proof}
\epsfxsize=5cm \epsfbox{gray3.eps}
\end{slide}

\begin{slide}{Bounds}
\begin{itemize}
\item
{\epsfxsize=4cm \epsfbox{BWE.eps}}
\item
$k=|B|=|W|=n-1+(n-2)(n+1)/4$ and $l=2+(n-2)(n-1)/2$
\item
$2^{2^{n^2/2\,-O(n)}} \leq$  \#games $\leq 2^{2^{n^2 \log 3+\log\log n + O(1)}}$
\end{itemize}
\end{slide}

\begin{slide}{number $N$ of 19x19 games}
\begin{itemize}
\item
$(103!)^{2^{154}} \leq N \leq 361^{0.012\cdot 3^{361}}$
\item in binary: $2^{2^{163}} < N <2^{2^{569}}$
\item in decimal: $10^{10^{48}} < N <10^{10^{171}}$
\end{itemize}
\end{slide}

\begin{slide}{In 1 dimension}
\begin{minipage}{5cm}
\begin{itemize}
\item stone at $i$ has weight $2^i$
\item predecessor: replace leftmost stone at $i$
by opposite stones at $0,1,\ldots,i-1$
\item $2^{n-2}$ skippable positions on each max path
\item \#games $\geq 2^{2^{n-1}}$
\end{itemize}
\end{minipage}
\begin{minipage}{6cm}
{\epsfxsize=6cm \epsfbox{leftcap.eps}}
\end{minipage}
\end{slide}

\overlays{5}{
\begin{slide}{Hamiltonian Games}
\begin{itemstep}
\item Games in which every legal position occurs.
\item Only one-dimensional boards can be Hamiltonian.
\item Equivalently, $G(1,n)$ must have a directed
Hamiltonian path starting at the empty position.
\item True for $n=1,3,4,5,6,7$.
\item Conjecture: true for all larger $n$ as well.
\end{itemstep}
\end{slide}
}

\overlays{6}{
\begin{slide}{Open problems}
\begin{itemstep}
\item Prove $L(m,n) \approx \alpha \beta^{m+n} L^{mn}$.
\item Prove Hamiltonicity of all $G(1,n), n \geq 3$.
\item Find more efficient algorithm for computing $L(m,n)$
\item Compute $L(19,19)$, the number of
      legal positions on a standard size Go board.
\item 363324268018 counters require about 10TB storage \\
      and years of computation time on a dedicated cluster.
\item write a decent playing Go program:-)
\end{itemstep}
\end{slide}
}

\end{document}
